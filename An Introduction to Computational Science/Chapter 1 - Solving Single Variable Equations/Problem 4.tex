\documentclass{article}

\usepackage{graphicx}
\usepackage{subcaption}
\usepackage{float}
\usepackage{enumitem}
\usepackage{amsmath}
\usepackage{amsthm}
\usepackage{booktabs}
\usepackage{amssymb}
\usepackage{geometry}
\usepackage{mathrsfs}
\usepackage{centernot}
\renewcommand{\qedsymbol}{$\blacksquare$}

\geometry{letterpaper, margin = 1in}

\begin{document}
	
	\begin{flushleft}	
	
		Brandon M. Keltz\\
		An Introduction to Computational Science by Allen Holder and Joseph Eichholz\\
		Chapter 1 - Solving Single Variable Equations\\
		September 7, 2019\\\
		
		\textit{Problem 4}. Show that the sequence defined by the recursive statement $d_{k + 1} = \frac{d_k}{2}$, with $d_1 = a$ converges with order 1 for any constant $a$. What does this say about the bisection method, especially if compared to Newton's method? \\\
		
		\begin{proof}
			
			Notice that the sequence
			\begin{align*}
				d_{k + 1} = \frac{d_k}{2},
			\end{align*}					
			where $d_1 = a$ has a closed-form solution
			\begin{align*}
				d_{k + 1} = \frac{a}{2^k}.
			\end{align*}
			Linear convergence of a sequence guarantees that
			\begin{align*}
				\lim_{k \to \infty} \frac{\left| x^{k + 1} - x^* \right|}{\left| x^k - x^* \right|^p} < \infty,
			\end{align*}
			where $p = 1$. Without loss of generality, consider $a$ nonzero. From the closed-form solution we have
			\begin{align*}
				\lim_{k \to \infty} \frac{\left| x^{k + 1} - x^* \right|}{\left| x^k - x^* \right|} = \lim_{k \to \infty} \frac{\left| a 2^{-k} - 0 \right|}{\left| a 2^{1 - k} - 0 \right|} = \lim_{k \to \infty} \frac{\left| a 2^{-k} \right|}{\left| a 2^{1 - k} \right|} = \lim_{k \to \infty} \frac{\left| 2^{-k} \right|}{\left| 2^{1 - k} \right|} = \lim_{k \to \infty} \frac{1}{2} = \frac{1}{2} < \infty,
			\end{align*}
			which gives us order 1 convergence for when $a$ is nonzero. 
			
		\end{proof}
		
		This shows the bisection method has a linear convergence. This comes from the iteration limit based on the bracketed interval we have
		\begin{align*}
			\varepsilon \geq \frac{b - a}{2^k},
		\end{align*}
		where $k$ is the number of iterations. The same analysis can be done since the form of the sequence and the error are equivalent to a constant. However, Newton's method has a quadratic convergence, which means that the bisection and Newton's method for this function would have the same order of convergence.
	
	\end{flushleft}
	
\end{document}